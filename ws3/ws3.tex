
\documentclass{article}
    \usepackage[utf8]{inputenc}
    \usepackage{comment}
    \usepackage{environ}
    \usepackage{xcolor}
    \usepackage[letterpaper, portrait, margin=0.5in]{geometry}
    \usepackage{amsmath}
    \usepackage{amssymb}
    
    \newif\ifshowsolutions
    
    \NewEnviron{solution}{
        \ifshowsolutions
        \ \\ 
        \ \\
        \textcolor{blue}{\smallskip \textbf{Solution:} \BODY}
        \else
        \ \\
        \ \\
        \fi}
    
    % comment the following line to hide solutions
    % \showsolutionstrue
    
    \begin{document}
    
    \part*{CS 70: RSA, Polynomial Applications, Un\_\_tability}
    \vspace{-7pt}
    \hrule
    \vspace{7pt}
    
    \section{RSA}
    \begin{enumerate}
        \item Suppose Alice wishes to send Bob a confidential message using RSA. For this, Bob must first set up his public-private key pair. Below, we show the choices Bob made in picking his keys, where he makes at least one mistake.
        
        Suppose that Bob chooses primes p = 7, q = 13. (Assume these are large enough.)
        
        He computes N = pq = 91.
        
        Then Bob chooses e = 3 so his public key is (3, 91).

        Finally Bob chooses d = 61 which is his private key.

        What mistake did Bob make?
        \begin{solution}
        \end{solution}
    \end{enumerate}
    \section{Polynomials and (mod p)}
    \begin{enumerate}
        \item Why do we use a prime p as a modulus so often?
        \begin{solution}
        \end{solution}
        \item Is the polynomial $3 x^3 + 5x^2 + 4x + 2$ perfectly divisible by $x-3 \ (\textnormal{mod } 7)$?
        \begin{solution}
        \end{solution}
        \item For $k < 7$, How many unique degree-$k$ polynomials are there mod $7$? For $k < p$, Mod $p$?
        \begin{solution}
        \end{solution}
        \item Suppose $x=4 \textnormal{ (mod } 7 ) $ and $x=7 \textnormal{ (mod }11) $. What is $x \textnormal{ (mod }77)$?
        \begin{solution}
        \end{solution}
        \item Let $r^2 = 1 \textnormal{ (mod } n)$. Show that if $r-1$ and $n$ are relatively prime, then $r = n - 1 \textnormal{ (mod } n)$.
        \begin{solution}
        \end{solution}
        \item Let $x_1, \ldots x_n$ be integers, and $p$ a prime number. Show that $\left(x_1 + \ldots + x_n\right)^p = x_1^p + \ldots + x_n^p \textnormal{ (mod } p)$.
        \begin{solution}
        \end{solution}
        \item Consider a two-variable polynomial $Z(x,y)=P(x)Q(y)$ modulo a prime p where $P(x)$ and $Q(y)$ are nonzero degree-$d$ polynomials where $d < p$. What is the maximum number of distinct pairs of $(i, j)$ that satisfy $Z(i, j) = 0 \textnormal{ (mod } p)$?
        \begin{solution}
        \end{solution}
    \end{enumerate}

    \section{Polynomial Applications}
    \begin{enumerate}
        \item Assume we send $n$ packets to Alice, and we know that $p = 20\%$ of any packets we sent are lost. How many packets should we send under standard error correcting schemes
        to ensure Alice can recover our message? What happens if $p=0.9$?
        \begin{solution}
        \end{solution}
        \item Given the error polynomial from Berlekamp-Welch algorithm, $x^2 + 3x + 2 \textnormal{ (mod } 11)$, for what ’x’ values are the points corrupted?
        \begin{solution}
        \end{solution}
        \item We'll prove how Berlekamp-Welch can work, and maybe provide some reason for why it needs $n+2k$ points. Alice encoded her $n$-length message in $P$, then sent
        $n+2k$ packets $[P(1) = x_1], [P(2) = x_2], [P(n+2k)=x_{n+2k}]$ over to us, of which at most $k$ have been corrupted.
        \begin{enumerate}
            \item What degree polynomial $P$ did Alice use to send us the message?
            \begin{solution}
            \end{solution}
            \item Let's call $R$ the polynomial made up of the $n+2k$ packets we received $[R(1) = r_1], [R(2) = r_2], [R(n+2k)=r_{n+2k}]$.
            With at most $k$ corruptions, for how many points among $1, \ldots, n+2k$ must $P$ and $R$ agree on (i.e. $P(x) = R(x)$?
            \begin{solution}
            \end{solution}
            \item Now assume we have two polynomials of degree part a) that both agree with $P$ on part b) number of points. How many points must these
            two polynomials agree on? What can we conclude from this?
            \begin{solution}
            \end{solution}
        \end{enumerate}
        \item The standard polynomial secret sharing is being used and we are working mod 5. Three shares are required to determine the secret, encoded as P(0). We have the following shares: P(1) = 2, P(2) = 0, P(3) = 2. What is the secret?
        \begin{solution}
        \end{solution}
        \item How might we split a secret up among $n$ people that requires two numbers? The last problem encoded the secret as $P(0)$, but what if one number is not
        enough to describe our secret?
        \begin{solution}
        \end{solution}
    \end{enumerate}
    \section{Un(coun,compu)tability}
    \begin{enumerate}
        \item Is the powerset of $\mathbb{N}$ countable (the set of all subsets of $\mathbb{N}$)? How would you prove this?
        \begin{solution}
        \end{solution}
        \item Are the integers $\mathbb{Z}$ countable? How about pairs of integers where one of the pair must be zero?
        \begin{solution}
        \end{solution}
        \item Is a countable union of countable subsets countable? This means $\bigcup_{i} U_i$ where $i \in \mathbb{N}$.
        \begin{solution}
        \end{solution}
        \item Consider the following program: 
\begin{verbatim}        
def is_mod_2(P):
    if (P implements the mod 2 function):
        return True 
    else:
        return False
\end{verbatim}
Show it cannot exist as a program.

(\textit{Hint}: Assume it exists, and show that it solves the halting problem. Because a program to solve the halting problem doesn't exist, neither can this one!)
\begin{solution}
\end{solution}
    \item Show that there exist numbers in $\mathbb{R}$ that cannot be computed. (Wow!!!)
    \end{enumerate}
    \end{document}
        