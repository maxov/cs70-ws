
\documentclass{article}
    \usepackage[utf8]{inputenc}
    \usepackage{comment}
    \usepackage{environ}
    \usepackage{xcolor}
    \usepackage[letterpaper, portrait, margin=0.5in]{geometry}
    \usepackage{amsmath}
    \usepackage{amssymb}
    
    \newif\ifshowsolutions
    
    \NewEnviron{solution}{
        \ifshowsolutions
        \ \\ 
        \ \\
        \textcolor{blue}{\smallskip \textbf{Solution:} \BODY}
        \else
        \ \\
        \ \\
        \fi}
    
    % comment the following line to hide solutions
    \showsolutionstrue
    
    \begin{document}
    
    \part*{CS 70: RSA, Polynomial Applications, Un\_\_tability}
    \vspace{-7pt}
    \hrule
    \vspace{7pt}
    
    \section{RSA}
    \begin{enumerate}
        \item Suppose Alice wishes to send Bob a confidential message using RSA. For this, Bob must first set up his public-private key pair. Below, we show the choices Bob made in picking his keys, where he makes at least one mistake.
        
        Suppose that Bob chooses primes p = 7, q = 13. (Assume these are large enough.)
        
        He computes N = pq = 91.
        
        Then Bob chooses e = 3 so his public key is (3, 91).

        Finally Bob chooses d = 61 which is his private key.

        What mistake did Bob make?
        \begin{solution}
            In RSA, we define $d = e^{-1} \textnormal{ (mod } (p-1)(q-1))$. This forces $e$ to be relatively prime to $(p-1)(q-1)$. But if $e = 3$ in this case,
            because $(p-1)(q-1) = 12 * 6 = 72$, 3 is not relatively prime to 72! So Bob needs to pick a better $e$.
        \end{solution}
    \end{enumerate}
    \section{Polynomials and (mod p)}
    \begin{enumerate}
        \item Why do we use a prime p as a modulus so often?
        \begin{solution}
        Working modulo $p$ has many advantages, stemming from the fact that primes have very few factors. We get
        the property that $ab = 0 \textnormal{ (mod } p) \implies a = 0 \textnormal{ (mod } p)$ or $b = 0 \textnormal{ (mod } p)$. In addition, every number that isn't congruent
        to zero has an inverse. This makes it `look' a lot like the real numbers or the rationals. This isn't the case for example if our modulus is $n = 12$, because $3 \cdot 4 = 0 \textnormal{ (mod } 12)$,

        \end{solution}
        \item Is the polynomial $3 x^3 + 5x^2 + 4x + 2$ perfectly divisible by $x-3 \ (\textnormal{mod } 7)$?
        \begin{solution}
        Yes it is! When doing polynomial divison, after you've finished a single step of multiplying, make sure you take all coefficients mod 7. Then you should find that
        \[
            3 x^3 + 5x^2 + 4x + 2 = (x - 3) (3x^2 + 4)
        \]
        \end{solution}
        \item For $k < 7$, How many unique degree-$k$ polynomials are there mod $7$? For $k < p$, Mod $p$?
        \begin{solution}
            A degree-$k$ polynomial is completely determined by $k+1$ points, for example when the polynomial is evaluated at
            $x = 0, 1, 2, \ldots k$. There are 7 choices for each of these points, so we get $7^{k+1}$. In general, we get $p^{k+1}$.
            There are 
        \end{solution}
        \item Suppose $x=4 \textnormal{ (mod } 7 ) $ and $x=7 \textnormal{ (mod }11) $. What is $x \textnormal{ (mod }77)$?
        \begin{solution}
            The CRT does apply to this problem, but it doesn't help us find a solution (at least what we learned in class). To figure out
            a number, just enumerate all numbers less than $77$ that are congruent to $7$ mod 11, and figure out which of those is congruent to 4 mod 11.
            18 is a number that comes up relatively early that satisfies this, but CRT does tell us that it is the unique number less than 77 that satistfiest
            these two congruences.
        \end{solution}
        \item Let $r^2 = 1 \textnormal{ (mod } n)$. Show that if $r-1$ and $n$ are relatively prime, then $r = n - 1 \textnormal{ (mod } n)$.
        \begin{solution}
            If $r-1$ and $n$ are relatively prime, then there must be $x$ and $y$ such that $x(r-1) + yn = 1$. Multiplying both sides by $r+1$, we have that $x(r^2 - 1) + yn (r + 1) = (r + 1)$. If we take
            both sides mod n, then we have $0 = r + 1 \textnormal{ (mod } n )$, which means $r = -1 = n -1 \textnormal{ (mod } n )$.
        \end{solution}
        \item Let $x_1, \ldots x_n$ be integers, and $p$ a prime number. Show that $\left(x_1 + \ldots + x_n\right)^p = x_1^p + \ldots + x_n^p \textnormal{ (mod } p)$.
        \begin{solution}
            Applying Fermat's last theorem on the left, we have that $(x_1 + \ldots + x_n)^p = x_1 + \ldots + x_n$. Applying it to each term on the right, we have that
            $x_1^p + \ldots + x_n^p = x_1 + \ldots + x_n$. Since these are the same, the two sides must be equal.
        \end{solution}
        \item Consider a two-variable polynomial $Z(x,y)=P(x)Q(y)$ modulo a prime p where $P(x)$ and $Q(y)$ are nonzero degree-$d$ polynomials where $d < p$. What is the maximum number of distinct pairs of $(i, j)$ that satisfy $Z(i, j) = 0 \textnormal{ (mod } p)$?
        \begin{solution}
            Since we're working modulo a prime, we have that $Z(x, y) = 0$ only if $P(x) = 0$ or $Q(y) = 0$, or perhaps both. Both $P$ and $Q$ can have at most $d$ distinct roots each, because they have degree at most $d$ (this is just a fact of polynomials). Let the roots of
            $P$ be $x_1, x_2, \ldots x_d$ and the roots of $Q$ be $y_1, y_2, \ldots y_d$.
            
            Then notice that $Z(x_1, 0) = Z(x_1, 1) = Z(x_1, 2) = 0$, in other words, $Z(x_i, n)$ is zero for any $x_i$ and $n$! There are
            $d$ choices for one of the $x_i$ and $p$ choices for the $n$, so there are $pd$ ways this can happen. Something similar happens lookig at $Z(n, y_i)$. So if we add up all these pairs, we get $2pd$.
            But notice that we double counted pairs like $Z(x_i, y_i)$, as these look like both $Z(x_i, n)$ and $Z(n, y_i)$. There are exactly $d^2$ of these, so subtract these from $2pd$ to get the answer.
            
            At the end, we get at most $2pd - d^2$ places where the polynomial can be zero.
        \end{solution}
    \end{enumerate}

    \section{Polynomial Applications}
    \begin{enumerate}
        \item Assume we send $n$ packets to Alice, and we know that $p = 20\%$ of any packets we sent are lost. How many packets should we send under standard error correcting schemes
        to ensure Alice can recover our message? What happens if $p=0.9$?
        \begin{solution}
            We must consider what happens \textit{after} we send the packets! If we only send $20\%$ extra packets, we will lose 20\% of the original packets we send AND 20\% of the 20\% additional packets we send, meaning we receive
            $\frac{24}{25}n$ uncorrupted packets in total, which isn't enough.
            So in the end, we want $(n+k)(1 - \alpha) \geq n$ to ensure the number of uncorrupted packets received is at least $n$.
            By doing some algebraic manipulation we find that 
            \[ k \geq n \frac{\alpha}{1 - \alpha}\].
        \end{solution}
        \item Given the error polynomial from Berlekamp-Welch algorithm, $x^2 + 3x + 2 \textnormal{ (mod } 11)$, for what ’x’ values are the points corrupted?
        \begin{solution}
            Factor the polynomial as $(x + 1)(x+2)$. The polynomial is 0 exactly at the points that there are errors, which would be $x = -1, x = -2$, but since we are working mod 11 these are $x = 10, x = 9$.
        \end{solution}
        \item We'll prove how Berlekamp-Welch can work, and maybe provide some reason for why it needs $n+2k$ points. Alice encoded her $n$-length message in $P$, then sent
        $n+2k$ packets $[P(1) = x_1], [P(2) = x_2], [P(n+2k)=x_{n+2k}]$ over to us, of which at most $k$ have been corrupted.
        \begin{enumerate}
            \item What degree polynomial $P$ did Alice use to send us the message?
            \begin{solution}
                The polynomial Alice sent to us encoded $n$ messages, so it has degree $n-1$.
            \end{solution}
            \item Let's call $R$ the polynomial made up of the $n+2k$ packets we received $[R(1) = r_1], [R(2) = r_2], [R(n+2k)=r_{n+2k}]$.
            With at most $k$ corruptions, for how many points among $1, \ldots, n+2k$ must $P$ and $R$ agree on (i.e. $P(x) = R(x)$)?
            \begin{solution}
                They must agree on at least $n+k$ points. This is because if there are most $k$ points that are different, there must be at least $(n + 2k) - k = n+k$ points that are the same.
            \end{solution}
            \item Now assume we have two polynomials of degree part a) that both agree with $P$ on part b) number of points. How many points must these
            two polynomials agree on? What can we conclude from this?
            \begin{solution}
                Let these two polynomials be $R_1$ and $R_2$. You can think of this situation as $R_1$ and $R_2$ being two possible messages we can receive from Alice that have been corrupted differently.
                If $R_1$ only differs from $P$ on $k$ points and $P$ differs from $R_2$ on $k$ points, this means that $R_1$ and $R_2$ differ from \textit{each other} on at most $2k$ points. Then they must
                share $n$ points! Because a degree $n-1$ polynomial is completely determined by those $n$ points, there can only be one degree $n-1$ polynomial that agrees with at least $n+k$ points of $R_1$ and at least $n+k$ points of $R_2$.
            \end{solution}
        \end{enumerate}
        \item The standard polynomial secret sharing is being used and we are working mod 5. Three shares are required to determine the secret, encoded as P(0). We have the following shares: P(1) = 2, P(2) = 0, P(3) = 2. What is the secret?
        \begin{solution}
            If we require 3 shares, that makes the polynomial degree 2.
            Setup a system of equations for $P(x) = p_0 + p_1 x + p_2 x^2$ by plugging in all the values:
            \[
                p_0 + p_1 + p_2 = 2
            \]
            \[
                p_0 + 2 p_1 + 4 p_2 = 0
            \]
            \[
                p_0 + 3 p_1 + 9 p_2 = 2
            \]
            We end up with $p_0 = 3$, $p_1 = 2$, $p_2 = 2$, so $P(x) = 3 + 2 x + 2 x^2$, so $P(0) = 3$ and the secret is $3$.
        \end{solution}
        \item How might we split a secret up among $n$ people that requires two numbers? The last problem encoded the secret as $P(0)$, but what if one number is not
        enough to describe our secret?
        \begin{solution}
            This problem is written a little badly, but the idea is if we have a secret that is actually a pair $(a, b)$, just design a degree $n-1$ polynomial such that
            $P(0) = a$, $P(1) = b$, by some method of interpolation. If enough people get together to reconstruct the polynomial, then they can recover both $a$ and $b$ by
            evaluating the polynomial at both points.
        \end{solution}
    \end{enumerate}
    \section{Un(coun,compu)tability}
    \begin{enumerate}
        \item Is the powerset of $\mathbb{N}$ countable (the set of all subsets of $\mathbb{N}$)? How would you prove this?
        \begin{solution}
            See the next worksheet
        \end{solution}
        \item Are the integers $\mathbb{Z}$ countable? How about pairs of integers where one of the pair must be zero?
        \begin{solution}
            See the next worksheet
        \end{solution}
        \item Is a countable union of countable subsets countable? This means $\bigcup_{i} U_i$ where $i \in \mathbb{N}$.
        \begin{solution}
            See the next worksheet
        \end{solution}
        \item Consider the following program: 
\begin{verbatim}        
def is_mod_2(P):
    if (P implements the mod 2 function):
        return True 
    else:
        return False
\end{verbatim}
Show it cannot exist as a program.

(\textit{Hint}: Assume it exists, and show that it solves the halting problem. Because a program to solve the halting problem doesn't exist, neither can this one!)
\begin{solution}
    See the next worksheet
\end{solution}
    \item Show that there exist numbers in $\mathbb{R}$ that cannot be computed. (Wow!!!)
    \begin{solution}
        See the next worksheet
    \end{solution}
    \end{enumerate}
    \end{document}
        