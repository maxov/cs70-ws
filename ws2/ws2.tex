
\documentclass{article}
\usepackage[utf8]{inputenc}
\usepackage{comment}
\usepackage{environ}
\usepackage{xcolor}
\usepackage[letterpaper, portrait, margin=0.5in]{geometry}

\newif\ifshowsolutions

\NewEnviron{solution}{
    \ifshowsolutions
    \ \\ 
    \ \\
    \textcolor{blue}{\smallskip \textbf{Solution:} \BODY}
    \else
    \ \\
    \ \\
    \fi}

% comment the following line to hide solutions
% \showsolutionstrue

\begin{document}

\part*{CS 70 Worksheet: Midterm Review}
\vspace{-7pt}
\hrule
\vspace{7pt}

\section{Functions and Modular Arithmetic}
\begin{enumerate}
    \item Consider $f(x) = 3x \pmod 5$. Is this function bijective?
    \begin{solution}
    \end{solution}
    \item Consider $f(x) = 4x \pmod {14}$. Is this function bijective?
    \begin{solution}
    \end{solution}
    \item Take any modulus $n$. For $a \in \{ 1 \ldots n - 1 \}$, when is $f(x) = ax$ bijective? When does $a$ have an inverse in general?
    \begin{solution}
    \end{solution}
    \item If $a \cdot b \equiv 0 \pmod p$ for $p$ prime, what can we say about $a$ and $b$?
    \begin{solution}
    \end{solution}
    \item If $f(x) = ax$ is bijective mod $n$, what can we say about $1 \cdot 2 \cdot \ldots (n - 1)$ and $1 \cdot 2 \cdot \ldots (n - 1) \cdot a^{n-1}$.
    \begin{solution}
    \end{solution}
    \item Prove Fermat's little theorem: If $p$ is a prime number, and $a \in \{ 1 \ldots p - 1 \}$, then $a^{p-1} \equiv 1 \pmod p$..
    \ \\
    \ \\
    \begin{solution}
    \end{solution}
    \item The integer a is a quadratic residue of n if $gcd(a, n) = 1$ and $x^2 \equiv a \pmod n$ has a solution.
    Prove that if p is prime, $p > 2$, then there are $\frac{p-1}{2}$ quadratic residues of p among $\{ 1, \ldots, p - 1 \}$.
    \ \\
    \begin{solution}
    \end{solution}
\end{enumerate}

\section{Proofs and Previous Problems}
\begin{enumerate}
    \item Prove that if $ n^2 \not\equiv 1 \pmod 7$, then $n \not\equiv 1 \pmod 7$.
    \ \\
    \begin{solution}
    \end{solution}
    \item Use induction to prove that $1 + \frac{1}{2} + \ldots + \left(\frac{1}{2}\right)^n < 2$ for all $n$. (Hint: strengthen the inductive hypothesis)
    \ \\
    \begin{solution}
    \end{solution}
    \item Show that every natural number has a prime factorization.
    \ \\
    \begin{solution}
    \end{solution}
    \item Consider a fully connected directed graph where between every pair of vertices $u$, $v$, either $u$ has an edge to $v$ or $v$ is has an edge to $u$. Prove that every vertex must be connected directly to the vertex with maximum in-degree with a path of length at most 2.
    \ \\ 
    \begin{solution}
    \end{solution}
    \item Consider a graph with $n$ vertices that has a path of the form $v_1$ \ldots $v_n$ involving all the vertices, and where $v_1$ and $v_2$ both have degree at least $n/2$. Show that the graph has a hamiltonian cycle (a cycle that visits every node in the graph at least once).
    \ \\ 
    \begin{solution}
    \end{solution}
\end{enumerate}

\section{F2017 \#6}
\begin{enumerate}
    \item Argue that any directed simple graph where every vertex has out-degree at least one has a directed cycle.
    \ \\
    \begin{solution}
    \end{solution}
    \item Consider the graph formed with vertices corresponding to the men and women in a stable marriage instance and edges according to two different stable pairings, S and S'. If a pair is in both pairings only include a single edge in G, which ensures it is a simple graph. Argue that there is a cycle of length strictly greater than 2.
    \ \\
    \begin{solution}
    \end{solution}
    \item Define a man m as feeling threatened by another man $m'$ with respect to a pairing S if (m,w) is in S and w likes $m'$ better than m. We define the male feeling threatened graph for a stable pairing S as the directed graph whose vertices are men and with a directed arc for each pair $(m, m')$ where m is feeling threatened by $m'$. Show that the male feeling threatened graph for the male optimal pairing has a cycle if there is more than one pairing. (Hint: using the previous parts may be helpful.)
    \ \\ 
\end{enumerate}

\end{document}
    