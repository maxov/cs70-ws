
\documentclass{article}
\usepackage[utf8]{inputenc}
\usepackage{comment}
\usepackage{environ}
\usepackage{listings}
\usepackage{xcolor}
\usepackage[letterpaper, portrait, margin=0.5in]{geometry}
\usepackage{amsmath}
\usepackage{amssymb}

\newif\ifshowsolutions

% comment the following line to hide solutions
\showsolutionstrue

\ifshowsolutions
    \newenvironment{solution}{

        \color{blue} \smallskip \textbf{Solution:}

    }{}
\else
    \NewEnviron{solution} {
        \ \\
        \ \\
    }
\fi

\begin{document}

\part*{CS 70: Un\_\_tability, Counting, Probability}
\vspace{-7pt}
\hrule
\vspace{7pt}

\section{Uncountability}
\begin{enumerate}
    \item Is the powerset of $\mathbb{N}$ countable (the set of all subsets of $\mathbb{N}$)? How would you prove this?
    \begin{solution}
        No, do a proof by diagonalization. Assume that the powerset is countable, then we must have an infinite listing of subsets of $\mathbb{N}$: $S_0, S_1, \ldots$.

        Define a subset of the naturals $S'$ by letting $x \in S'$ iff $x \notin S_x$. This is basically inverting the diagonal. Then this subset has some index $i$ in the listing,
        but $i \in S'$ iff $i \notin S_i$ iff $i \notin S'$, which is always a contradiction.
    \end{solution}
    \item Are the integers $\mathbb{Z}$ countable? How about pairs of integers $(x, y)$ where $x = 0$ or $y = 0$?
    \begin{solution}
        Yes, see the proof presented in lecture. This constructs some bijective function that maps the even integers to positive integers and odd
        integers to negative ones.

        There is some similar proof for the pairs with one coordinate zero, except now working mod 4.
    \end{solution}
    \item Is a countable union of countable subsets countable? This means $\bigcup_{i} U_i$ where $i \in \mathbb{N}$.
    \begin{solution}
        Yes. A countable union ``looks like'' $\mathbb{N} \times \mathbb{N}$, let $(x, y)$ map to the $y$th element of $U_x$, then
        this is a valid way to count countable unions.
    \end{solution}
    \item Is the set of all irrational numbers countable?
    \begin{solution}
        No. If it was, then the reals would be a union of two countable sets (the rationals and irrationals) and be countable. This is a contradiction,
        because as proved in class, the reals aren't countable.
    \end{solution}
    \item Is the set of all programs countable?
    \begin{solution}
        Yes. There are many ways to encode programs with numbers, one of them is ASCII, like giving each letter in the program a bit. But the important thing is we can give at least
        one number for each possible program.
    \end{solution}
    \item Show that there are numbers in $\mathbb{R}$ that cannot be computed. (Wow!!!)
    \begin{solution}
        We showed the set of programs is countable. The set of inputs is countable too, so the set of all pairs of programs and inputs we could run them
        on $(P, x)$ is also countable. Then, since each pair $(P, x)$ corresponds to one possible output the program could give, the set of all possible outputs
        all possible programs could give on all possible inputs is still countable! But inside the reals which are strictly bigger than countable sets,
        this means we have real numbers that could not be computed by any program! Intuitively, these have to be `infinite' objects with an infinite amount of information
        that a program would never give to us in finite time, like $\pi$.
    \end{solution}
\end{enumerate}
\section{Uncomputability}
\begin{enumerate}
    \item Consider the following program: 
    \begin{lstlisting}[language=Python] 
def is_mod_2(P):
    if (P implements the mod 2 function):
        return True 
    else:
        return False
    \end{lstlisting}
    Show it cannot exist as a program.
    \begin{solution}
    Assume that it does, we can then solve the halting problem with a program! We want to build halt so that it uses \texttt{is\_mod\_2}.
    
\begin{lstlisting}
def halt(P, x):
    def Q(y):
        P(x)
        return y % 2
    return is_mod_2(Q)
\end{lstlisting}

    We basically modify the program P into Q so that is\_mod can tell us whether P halts! Notice that if P halts, then running $P(x)$ will finish,
    so then Q implements the mod 2 function. But if P doesn't halt, then Q doesn't halt, so it can't implement the is\_mod\_2.

    \end{solution}
    \item Consider this program:
    \begin{lstlisting}[language=Python]
def returns_42_on_42(P):
    x = P(42)
    if x = 42:
        return True
    else:
        return False
    \end{lstlisting}
    Can this exist? What if we replace the if condition with \texttt{if P(42) eventually halts and gives us 42}?
    \begin{solution}
    Yes it can! We just wrote it. :) 
    More intuitively, this program calls P, meaning that it doesn't really do any ``magic'' to analyze P and figure out if
    it halts or not. This is very different from the program that returns true if $P(42)$ eventually halts, because that one
    performs the magic to figure out that $P$ does eventually halt, and even if P does not halt, it still will.
    \end{solution}
\end{enumerate}

\section{Counting}
\begin{enumerate}
    \item How many permutations of SUPERMAN are there?
    \begin{solution}
        There are no repeated letters here, so we just have $8!$.
    \end{solution}
    \item How many for ARKANSAS?
    \begin{solution}
        We can start with $8!$, but this makes all the As and Ss mean different objects we are arranging, and we want to consider
        them as the same. We can divide by $3!$ to get rid of
        overcounting the As and divide by $2!$ to get rid of the Ss.
    \end{solution}
    \item We have 5 cookies we are trying to divide between 3 students. How many ways are there to divide the cookies among all the students?
    \begin{solution}
    This becomes a stars and bars problem. There are 2 bars, not 3 (because 2 bars makes 3 `compartments' to give people cookies), and 5 stars for each
    cookie. Then this is ${7 \choose 2}$, as we are just choosing the locations for the 2 bars which determines how many cookies each person gets.
    \end{solution}
    \item How many ways are there if we want to give every student at least one cookie?
    \begin{solution}
    First give each student a cookie! Now we are just trying to distribute 2 cookies among 3 students, ignoring the 3 cookies we just gave,
    which is ${4 \choose 2}$.
    \end{solution}

    \item Let $p, q$ be prime. How many numbers are there among $1, 2, \ldots (pq)^2$ that are relatively prime to $pq$?
    \begin{solution}
    This is an example of inclusion, exclusion. $p, 2p, \ldots, pqpq$ are not relatively prime to p, and $q, 2q, \ldots, pqpq$ are not relatively
    prime to $q$. That gives $pq^2$ numbers not relatively prime to $p$, and $qp^2$ not relatively prime to $q$. However, we have double counted the
    numbers that both $p$ and $q$ divide! These are $pq, 2pq, \ldots pqpq$, of which there are $pq$. So in total we have $pq^2 + qp^2 - pq$ numbers not
    relatively prime to $p$ or $q$. But to find the numbers that are, we subtract that from all the numbers, $pqpq$, so we end up with
    $pqpq + pq - pq^2 - qp^2$.
    \end{solution}

    \item How many combinations of even natural numbers $(x_1,x_2,x_3,x_4)$ are there such that
    $x_1 + x_2 + x_3 + x_4 = 20$?
    \begin{solution}
    This is a stars and bars in disguise! Notice that if all the numbers are even, for example $4, 4, 4, 6$, we can divide both sides
    of the equation to obtain $2 + 2 + 2 + 3 = 10$. So this is just counting how many sets of 4 non-negative natural numbers add up to 10, which is stars and bars.
    There are 10 stars and 4 compartments / 3 bars, so we have ${13 \choose 3}$.
    \end{solution}

    \item There is a class with 2n children where n are boys and n are girls. How many ways are there to arrange them in a line so that they alternate by gender?
    \begin{solution}
    We already know that the boys and girls alternate by gender, i.e. BGBGBG or GBGBGB. Then either of these sequences are completely
    determined by what order we put the boys in and what order we put the girls in, which is $(n!)^2$. That is only counting one of the
    possibilities though, i.e. we just counted how many ways we could have BGBGBG. Multiply by 2, so $2 (n!)^2$ to get the final answer.
    \end{solution}

    \item How many ways are there to arrange them where all the girls are before all the boys?
    \begin{solution}
    This is a similar case, the sequence is completely determined by how we arrange the girls independently from arranging the boys, so $(n!)^2$.
    \end{solution}

    \item How many ways are there to arrange them so that all the girls are in an uninterrupted block? (there is no boy in between two girls)
    \begin{solution}
    This is kind of like arranging $n+1$ things, where $n$ of them are boys, and one is the giant block of $n$ girls. There are $n+1$ ways to do this.
    Then it looks like the previous problems, now it just depends how we order the girls and boys within their respective genders, so we end up with
    $(n+1) \cdot n! \cdot n!$.
    \end{solution}

    \item How many ways are there for neither the girls nor the boys to stand in an uninterrupted block?
    \begin{solution}
    This is inclusion/exclusion. There are $(2n)!$ ways to arrange the girls and boys in general. So let's subtract from that
    the ways that either the girls or the boys stand in an uninterrupted block. Both of these are $(n+1) \cdot n! \cdot n!$.
    But we're double counting if the all the boys and all the girls are both uninterrupted, which is number 8, done twice for
    all the girls before all the boys and all the boys before all the girls, which is $2 (n!)^2$. Then the number of ways
    either the girls or boys stand in an uninterrupted block is $2 (n+1) \cdot (n!)^2 - 2 \cdot (n!)^2 = 2n \cdot (n!)^2$. Finally,
    we have to subtract that from $(2n)!$ to get $(2n)! - 2n \cdot (n!)^2$.
    \end{solution}

    \item Use a combinatorial argument to prove that $\sum_{k=0}^n {n \choose k}^2 = {2n \choose n}$.
    \begin{solution}
        ${n \choose k} = {n \choose n - k}$. So now let's try to show 

        \[
            \sum_{k=0}^n {n \choose k} {n \choose n -k } = {2n \choose n}
        \]

        The right hand side is just picking $n$ items from $2n$. For the left hand side, split the $2n$ set into two $n$-sized sets, call them the left and right sets.
        Then our strategy is to still pick $n$ items, where these items could come from either set. We first have to pick how many come from the left set,
        call that $k$. Then the number that come from the right set is $n-k$, so the number of ways to pick $n$ elements where $k$ are from the
        left and $n-k$ are from the right is ${n \choose k} {n \choose n-k}$.
    \end{solution}

    \item Give a combinatorial proof of ${k+n - 1 \choose n-1} = \sum_{i=0}^k {k-i+n-2 \choose n-2}$.
    \begin{solution}
    I'll omit the solution because I'll put this on the next worksheet :)
    \end{solution}

\end{enumerate}
\end{document}
    